\documentclass[]{politex}
% ========== Opções ==========
% pnumromarab - Numeração de páginas usando algarismos romanos na parte pré-textual e arábicos na parte textual
% abnttoc - Forçar paginação no sumário conforme ABNT (inclui "p." na frente das páginas)
% normalnum - Numeração contínua de figuras e tabelas 
%	(caso contrário, a numeração é reiniciada a cada capítulo)
% draftprint - Ajusta as margens para impressão de rascunhos
%	(reduz a margem interna)
% twosideprint - Ajusta as margens para impressão frente e verso
% capsec - Forçar letras maiúsculas no título das seções
% espacosimples - Documento usando espaçamento simples
% espacoduplo - Documento usando espaçamento duplo
%	(o padrão é usar espaçamento 1.5)
% times - Tenta usar a fonte Times New Roman para o corpo do texto
% noindentfirst - Não indenta o primeiro parágrafo dos capítulos/seções


% ========== Packages ==========
\usepackage[utf8]{inputenc}
\usepackage{amsmath,amsthm,amsfonts,amssymb}
\usepackage{graphicx,cite,enumerate}


% ========== Language options ==========
\usepackage[brazil]{babel}
%\usepackage[english]{babel}


% ========== ABNT (requer ABNTeX 2) ==========
%	http://www.ctan.org/tex-archive/macros/latex/contrib/abntex2
%\usepackage[num]{abntex2cite}

% Forçar o abntex2 a usar [ ] nas referências ao invés de ( )
%\citebrackets{[}{]}


% ========== Lorem ipsum ==========
\usepackage{blindtext}

% ========== Custom packages ========
\usepackage[hidelinks]{hyperref}
\usepackage{subcaption}
\usepackage{inconsolata}
\usepackage{listings}
\lstset{ %Personalizar a formatação de listagens de código
	tabsize=2,
	basicstyle=\ttfamily\linespread{0.8}\small 
}


% ========== Opções do documento ==========
% Título
\titulo{Projeto HomeSky}

% Autor
\autor{
	Fábio T. Muramatsu \\%
	Henrique Rodrigues\\%
	Ricardo B. Gallego}
	

% Para múltiplos autores (TCC)
%\autor{Nome Sobrenome\\%
%		Nome Sobrenome\\%
%		Nome Sobrenome}

% Orientador / Coorientador
\orientador{Reginaldo Arakaki}
\coorientador{Marcelo Pita e Leandro Souza}

% Tipo de documento
\tcc{Eletricista com Ênfase em Computação}
%\dissertacao{Engenharia Elétrica}
%\teseDOC{Engenharia Elétrica}
%\teseLD
%\memorialLD

% Departamento e área de concentração
\departamento{Departamento de Engenharia de Computação e Sistemas Digitais}
\areaConcentracao{Engenharia de Computação}

% Local
\local{São Paulo}

% Ano
\data{2016}




\begin{document}

\renewcommand{\lstlistingname}{Listagem} 
\renewcommand{\thelstlisting}{\arabic{lstlisting}}% Imprimir legenda como Listagem X

% ========== Capa e folhas de rosto ==========
{
	\fontfamily{phv}\selectfont
	\capa
}
\falsafolhaderosto
\folhaderosto


% ========== Folha de assinaturas (opcional) ==========
%\begin{folhadeaprovacao}
%	\assinatura{Prof.\ X}
%	\assinatura{Prof.\ Y}
%	\assinatura{Prof.\ Z}
%\end{folhadeaprovacao}


% ========== Ficha catalográfica ==========
% Fazer solicitação no site:
%	http://www.poli.usp.br/en/bibliotecas/servicos/catalogacao-na-publicacao.html


% ========== Dedicatória (opcional) ==========
%\dedicatoria{Dedicatória}


% ========== Agradecimentos ==========
%\begin{agradecimentos}

%Thanks...

%\end{agradecimentos}


% ========== Epígrafe (opcional) ==========
\epigrafe{%
	\emph{``Computing is not about computers any more. It is about living.''}
	\begin{flushright}
		{Nicholas Negroponte}
	\end{flushright}
	\emph{``Talk is cheap. Show me the code.''}
	\begin{flushright}
		{Linus Torvalds}
	\end{flushright}
	\emph{``Far too often, 'software engineering' is neither engineering nor about software.''}
	\begin{flushright}
		{Bjarne Stroustrup}
	\end{flushright}
}


% ========== Resumo ==========
\begin{resumo}
A popularização do conceito de automação residencial foi acompanhado pela disponibilização de diversas soluções comerciais, tais como o Apple HomeKit e o Samsung SmartThings. Tais soluções possuem limitações importantes, como a disponibilização de plataformas parcialmente abertas ou proprietárias e a ausência de um processo aprendizagem para automação residencial. A primeira limitação foi resolvida através do projeto do protocolo Rainfall, que funciona a nível de aplicação e possui especificação aberta. Esse protocolo foi implementado em forma de biblioteca e disponibilizado no serviço Github, podendo ser utilizado como base para o desenvolvimento de dispositivos de uma rede de sensores doméstica. O funcionamento do protocolo foi demonstrado através da prototipação de sensores, atuadores e controladores em computadores Raspberry Pi, que se conectavam com um servidor em nuvem e com um aplicativo móvel. Esta demonstração permitiu efetuar ações presentes nas soluções disponíveis comercialmente, tais como verificar o estado dos dispositivos, enviar comandos a eles e definir regras de automação manualmente, pelo aplicativo, de forma remota. A segunda limitação foi abordada através da implementação de uma prova de conceito para um problema específico de automação residencial, qual seja o controle de iluminação. Para tanto, foi desenvolvido um algoritmo baseado em indução por árvores de decisão, capaz de gerar regras a partir de leituras de sensores de iluminação e presença. O algoritmo foi testado em dados coletados nas residências dos membros do projeto, produzindo resultados satisfatórios.
%
\\[3\baselineskip]
%
\textbf{Palavras-Chave} -- Palavra, Palavra, Palavra, Palavra, Palavra.
\end{resumo}


% ========== Abstract ==========
%\begin{abstract}
%Abstract...
%%
%\\[3\baselineskip]
%%
%\textbf{Keywords} -- Word, Word, Word, Word, Word.
%\end{abstract}


% ========== Listas (opcional) ==========
\listadefiguras
\listadetabelas

% ========== Listas definidas pelo usuário (opcional) ==========
\begin{pretextualsection}{Lista de símbolos}

\bgroup
\def\arraystretch{1.2}%  1 is the default, change whatever you need
\begin{tabular}[h]{l p{0.8\textwidth}}
6loWPAN & \textit{IPv6 over Low-power Wireless Personal Area Network}\\
AFP & \textit{Adaptive Frequency Hopping}\\
API & \textit{Application Programming Interface}\\
Bluetooth LE & \textit{Bluetooth Low Energy}\\
CBOR & \textit{Concise Binary Object Representation}\\
CDMA & \textit{Code Division Multiple Access}\\
CoAP & \textit{Constrained Application Protocol}\\
CSMA-CA & \textit{Carrier Sense Multiple Access - Collision Avoidance}\\
DDS & \textit{Data Distribution Service}\\
DNF & \textit{Disjunctive Normal Form}\\
FFD & \textit{Full Function Device}\\
HTTP & \textit{Hypertext Transfer Protocol}\\
IEEE & Instituto de Engenheiros Eletricistas e Eletrônicos\\
IETF & \textit{Internet Engineering Task Force}\\
IoT & \textit{Internet of Things}\\
IP & \textit{Internet Protocol}\\
IPv6 & \textit{Internet Protocol version 6}\\
ISM & \textit{Industrial, Scientific and Medical}\\
ITU-T & \textit{International Telecommunication Union - Telecommunication Standardization Sector}\\
JSON & \textit{JavaScript Object Notation}\\
MQTT & \textit{Message Queue Telemetry Transport}\\
MTU & \textit{Maximum Transmission Unit}\\
OS & \textit{Operating System}\\
OSI & \textit{Open Systems Interconnection}\\
PIN & \textit{Personal Identification Number}\\
PIR & \textit{Passive Infrared}\\
REST & \textit{Representational State Transfer}\\
RFD & \textit{Reduced Function Device}\\
TCP & \textit{Transmission Control Protocol}\\
TDM & \textit{Time Division Multiplexing}\\
UDP & \textit{User Datagram Protocol}
\end{tabular}
\egroup

\end{pretextualsection}

% ========== Sumário ==========
\sumario



% ========== Elementos textuais ==========
	
\chapter{Introdução}\label{chp:intro}

\section{Apresentação}\label{sec:presentation}
O conceito de \textit{smart houses} (casas inteligentes) tem ganhado grande destaque no meio acadêmico e no mercado nos últimos anos, com o desenvolvimento de tecnologias interativas e de redes sem fio \cite{harper2006}. É intuitivo que a possibilidade de concretização desse conceito de casa inteligente, viabilizado pelo avanço de tais tecnologias, foram decisivos  para a sua popularização. Não surpreende, pois, que em 2015, a empresa de consultoria Gartner tenha destacado o item "casa conectada" em seu relatório anual de tendências de tecnologias emergentes, denominada \textit{Hype Cycle}, como pode ser visto na Figura \ref{fig:gartner}.

\begin{figure}[h]
	\centering
	\caption{Relatório \textit{Hype Cycle} da Gartner destacando a tecnologia de casas conectadas.}
  \includegraphics[width=0.9\textwidth]{imagens/gartner.png}
  \label{fig:gartner}
  
  Fonte: \cite{gartner}
\end{figure}

O conceito de casa inteligente é definido por \cite{jiang2004} como "uma residência incorporando uma rede de comunicações que conecta serviços e equipamentos elétricos, permitindo que eles sejam controlados remotamente, monitorados ou acessados". Esta definição explicita o caráter interativo e automatizado inerente ao conceito, abrindo uma margem muito grande de possíveis aplicações de utilidade social. Dentre tais aplicações, inclui-se prover automação residencial e conectividade social \cite{harper2006}, além de fornecer um ambiente seguro e monitorável a idosos ou deficientes \cite{chan2008}.


\section{Soluções existentes}\label{sec:solutions}
Acompanhando a popularização do conceito de \textit{smart houses}, diversas soluções e plataformas foram desenvolvidas e lançadas no mercado. Como exemplos, pode-se citar o Apple HomeKit \cite{homekit}, Wireless Sensor Tags \cite{wsensortags}, WigWag \cite{wigwag} e o Samsung SmartThings \cite{smartthings}.

Todas as soluções citadas, à exceção do Apple HomeKit, seguem uma arquitetura similar. Os sensores e atuadores distribuídos pela residência são conectados a um controlador central, responsável por coletar leituras dos sensores e comandar ações aos atuadores. Além disso, este controlador se conecta a um servidor em nuvem, que centraliza o armazenamento de dados e fornece uma interface aos usuários para monitorar e definir o estado da residência.

A solução desenvolvida pela Apple baseia-se na utilização dos dispositivos móveis da empresa para efetuar o monitoramento e controle mencionados. Assim, ela é fortemente dependente do ecossistema Apple para o funcionamento, especialmente pelo fato de o suporte ser dado somente pelo sistema operacional proprietário iOS. No entanto, a empresa permite o desenvolvimento de dispositivos de terceiros compatíveis com o HomeKit, através da disponibilização de um \textit{framework} próprio. Os produtos desenvolvidos devem ser certificados pela Apple, processo este envolvendo a obtenção de licenças e submissão a análises.

O Wireless Sensor Tags apresenta uma solução baseada em um controlador local (\textit{tag manager}) conectado à Internet, disponibilizando diversos sensores compatíveis com o controlador. Basicamente, ele permite a utilização de um aplicativo para monitorar os sensores, definindo alertas de acordo com as leituras obtidas. Não é mencionado suporte a atuadores, nem a possibilidade de desenvolver dispositivos de terceiros compatíveis com o sistema.

O WigWag também é uma solução baseada em um controlador local (aqui denominado \textit{relay}), podendo operar conectado à nuvem ou não. A interface disponível possibilita a definição de regras de automação, controlando atuadores caso determinadas leituras de sensores forem verificadas. O aspecto mais interessante desta solução é a capacidade de integrar dispositivos de terceiros utilizando uma plataforma que se auto-denomina de código  aberto, chamada deviceJS. No entanto, o acesso a esta plataforma se encontra fechado no momento de escrita deste relatório.

Por fim, de modo análogo às demais, a solução SmartThings baseia-se em um controlador local (\textit{hub}), que se conecta a servidores em nuvem próprios da Samsung. Esta solução destaca-se pelo foco dado à integração com dispositivos de terceiros, possuindo a documentação mais completa dentre os exemplos analisados. Tal integração baseia-se em dois componentes de software básicos: \textit{device handlers} e \textit{smart apps}. Os \textit{device handlers} funcionam como \textit{drivers}, sendo executados no controlador local. A função deles é traduzir comandos de alto nível do controlador (e.g., ligar a luz) para sinais de controle específicos do dispositivo. Os \textit{smart apps}, por sua vez, adicionam a parte de inteligência do sistema, permitindo a criação de regras de automação de modo similar ao WigWag. A empresa disponibiliza aos desenvolvedores um ambiente de desenvolvimento completo para implementar os componentes citados.

\section{Motivação e Objetivos}\label{sec:goals}
Conforme mencionado, o conceito de \textit{smart houses} tem o potencial de trazer vários benefícios aos usuários. Logo, é interessante incentivar o desenvolvimento de sistemas abertos, desvinculando-os de empresas e serviços específicos e tornando-os mais flexíveis para o usuário. Note que, das soluções apresentadas, nenhuma é genuinamente aberta. As soluções WigWag e Samsung SmartThings são as que mais se aproximam dessa ideologia ao disponibilizar plataformas em código aberto para integrar dispositivos de terceiros, mas ainda possuem protocolos de comunicação proprietários (ao menos não documentados) que são dependentes de controladores e servidores em nuvem proprietários.

O grupo acredita que a existência de um protocolo de comunicação que possibilite a troca de dados e a coordenação dos dispositivos da rede de sensores resulte em maior flexibilidade aos desenvolvedores desses equipamentos. Isso ocorre pois existiria uma garantia de que um dispositivo aderente ao protocolo funcione com quaisquer outros dispositivos ou controladores que também o respeitem, removendo quaisquer dependências com soluções ou fabricantes existentes.

Além disso, a inteligência contida nas \textit{smart houses} é razoavelmente limitada nas soluções existentes. Conforme apresentado anteriormente, toda a inteligência é provida pelo usuário, que configura manualmente regras envolvendo sensores e atuadores. Ou seja, uma casa que se autoconfigurasse ou sugerisse regras ou ações ao usuário teria destaque no mercado, eliminando ainda mais a intervenção do usuário e fornecendo assim uma  experiência altamente customizada.

Um último aspecto a se mencionar é o fato de nenhuma das soluções estar disponível no mercado brasileiro no momento de escrita deste relatório (início de 2016). Os equipamentos do SmartThings, por exemplo, nem sequer são enviados ao Brasil, estando, pois, indisponíveis para importação. Vê-se a necessidade, portanto, do desenvolvimento de uma solução nacional para o segmento das \textit{smart houses}.

Assim, pode-se dividir o objetivo do presente projeto nos seguintes itens:
\begin{enumerate}[\quad (i)]
	\item Projetar e implementar um \textbf{protocolo aberto de comunicação em nível de aplicação} para ser utilizada em uma rede de sensores sem fio voltada à automação residencial, e
	\item Projetar e implementar um \textbf{algoritmo de aprendizagem de máquina ou mineração de dados} de modo a prover automação residencial. No presente trabalho, será dado enfoque ao desenvolvimento de um algoritmo no domínio de controle de iluminação.
\end{enumerate}

Cabe ressaltar desde já que este trabalho não será focado nos aspectos de segurança do protocolo, como detalham as seções de não-escopo deste documento. O foco dado é em prover as funcionalidades básicas de uma rede de sensores e atuadores doméstica, permitindo a interoperabilidade entre os dispositivos.

\section{Organização do Documento}
O documento está organizado da seguinte forma. O capítulo \ref{chp:espec_metodologia} contém a especificação de requisitos, os não-escopos e a metodologia adotada no projeto; o capítulo \ref{chp:redesensores} trata do desenvolvimento do protocolo a ser executado na rede local de dispositivos, contemplando o objetivo (i); o capítulo \ref{chp:serviconuvem} aborda serviços em nuvem e o aplicativo móvel de interface com o usuário, incluindo-se o algoritmo de aprendizagem proposto no objetivo (ii); o capítulo \ref{chp:testes} descreve o procedimento de teste modular e de integração adotado no projeto, e o capítulo \ref{chp:conclusoes} contém uma análise crítica do grupo em relação aos resultados e ao preocesso de desenvolvimento deste projeto.
\chapter{Especifica��o}

\section{Divis�o do Projeto}
O projeto pode ser dividido em dois sub-projetos, cada qual lidando com um objetivo listado na se��o anterior. O primeiro sub-projeto tem a finalidade de estruturar uma \textbf{rede de sensores} e atuadores utilizando um protocolo aberto, conforme descreve o objetivo i. O segundo sub-projeto busca desenvolver um \textbf{servi�o em nuvem} que agregue os dados coletados pela rede de sensores, efetuando, entre outras atividades, o processo de aprendizagem descrito no objetivo ii.

O primeiro passo na especifica��o consiste na identifica��o dos \textit{stakeholders}. Em seguida, a especifica��o do projeto ser� feita por meio de requisitos funcionais e n�o-funcionais, que est�o listados em tr�s classes:
\begin{itemize}
	\item Requisitos de sistema, relativos � integra��o de ambos os sub-projetos;
	\item Requisitos da rede de sensores, referente ao sub-projeto dedicado a atingir o objetivo i; 
	\item Requisitos do servi�o em nuvem, referente ao sub-projeto dedicado a atingir o objetivo ii.
\end{itemize}

\section{Descri��o dos \textit{stakeholders}}
Os seguintes \textit{stakeholders} foram identificados no projeto:
\begin{itemize}
	\item Usu�rio final, cujo interesse seria adquirir e instalar dispositivos em sua resid�ncia, de modo a obter automa��o residencial. Um perfil poss�vel para este tipo de stakeholder seria uma pessoa com conhecimento t�cnico b�sico em computa��o e redes de computadores. Ele deve ser capaz de conectar os dispositivos na rede desejada (por exemplo, na rede Wi-Fi por onde todo o tr�fego deve passar), mas os demais procedimentos de estrutura��o da rede de sensores segundo o protocolo devem ser feitos de forma transparente;
	\item Desenvolvedor, cujo interesse seria estudar a documenta��o dos protocolos e algoritmos desenvolvidos, e projetar dispositivos compat�veis. Este stakeholder seria usu�rio das bibliotecas (APIs) resultantes do projeto. Ele espera encontrar documentos explicando o funcionamento dos protocolos e dos algoritmos desenvolvidos, bem como documenta��es e exemplos de uso das APIs que deve utilizar. Ainda, estes desenvolvedores consideram como diferencial o fato de a licen�a de uso das tecnologias disponibilizadas serem abertas e isentas de pagamento.
\end{itemize}

\section{Requisitos Funcionais} \label{sec:reqfunc}
\subsection{De sistema}
\begin{itemize}
	\item O sistema deve coletar dados do ambiente domiciliar e possibilitar a ativa��o de atuadores automaticamente ou por a��o humana;
	\item O sistema deve permitir que o usu�rio visualize o estado atual do sistema;
	\item O sistema deve se adaptar a altera��es de gostos e prefer�ncias do usu�rio.
\end{itemize}

\subsection{Da rede de sensores}
\begin{itemize}
	\item Toda a comunica��o deve ser realizada atrav�s de um protocolo �nico e formalizado;
	\item O protocolo de comunica��o deve permitir a identifica��o dos dispositivos da rede, bem como o envio de dados e de comandos de atua��o.
\end{itemize}

\subsection{Do servi�o em nuvem}
\begin{itemize}
	\item Permitir acesso remoto em qualquer localiza��o para que usu�rio controle seus dispositivos � dist�ncia;
	\item Aprender os h�bitos do usu�rio e configurar a rede de sensores automaticamente para realizar os seus gostos.

\end{itemize}

\section{Requisitos N�o-Funcionais} \label{sec:reqnfunc}
\subsection{De sistema}
\begin{itemize}
	\item F�cil instala��o de sensores e integra��o com a nuvem;
	\item Utilizar linguagens, ambientes e bibliotecas abertas.
\end{itemize}

\subsection{Da rede de sensores}
\begin{itemize}
	\item O protocolo de comunica��o desenvolvido deve ser aberto, de modo que qualquer fabricante possa criar um dispositivo que compat�vel com a rede;
	\item Deve possibilitar o uso de diversas tecnologias de comunica��o, tais como ZigBee, 802.15.4, e UDP;
	\item O protocolo deve gerar pacotes de tamanhos compat�veis com tecnologias de redes existentes para redes de sensores sem fio;
	\item A rede deve continuar operante mesmo sem acesso � nuvem.
\end{itemize}

\subsection{Do servi�o em nuvem}
\begin{itemize}
	\item Disponibilidade de 99\% do tempo; 
	\item API que possibilite o acesso atrav�s de aplica��es de terceiros/aplicativo/site.
\end{itemize}

\section{Arquitetura do Sistema} \label{sec:arquitetura}
Seguindo a descri��o apresentada do projeto, o grupo definiu a arquitetura do sistema de automa��o residencial conforme mostra a Figura \ref{fig:arquitetura}.

\begin{figure}[h]
	\centering
	\caption{Arquitetura do sistema de automa��o.}
  \includegraphics[width=0.8\textwidth]{imagens/arquitetura.jpg}
  \label{fig:arquitetura}  
\end{figure}

A arquitetura cont�m dispositivos sensores e atuadores, que efetuam a coleta de dados nas resid�ncia e atuam nelas. Esses dispositivos se conectam a um controlador local, que efetua o gerenciamento dos n�s locais (a aceita��o de n�s na rede, por exemplo), e a interface entre os dispositivos locais e a infraestrutura em nuvem. Tais dispositivos (excluindo a infraestrutura em nuvem) comp�em a rede local de sensores e atuadores, e ser�o o foco do primeiro sub-projeto.

Al�m disso, constam no diagrama um servidor em nuvem e um servi�o de \textit{analytics}. O servidor em nuvem atua como um agregador dos dados coletados pelos diversos controladores locais. Sobre este banco de dados, o servi�o de \textit{analytics} (minera��o de dados e aprendizagem de m�quina) atuar� no sentido de derivar padr�es de comportamento dos usu�rios, sugerindo regras de automa��o residencial. Por fim, aponta-se a exist�ncia de um elemento de interface gr�fica com a qual o usu�rio pode interagir com o sistema, representado na arquitetura apresentada como um \textit{smartphone}. O elemento de interface com o usu�rio se comunica com o servidor em nuvem, e pode eventualmente se comunicar com o controlador local. Tais componentes da arquitetura ser�o foco do segundo sub-projeto.
\input{texto/03redesensores.tex}
\chapter{Projeto do Servi�o em Nuvem}\label{chp:serviconuvem}
Este cap�tulo possui a finalidade de projetar e implementar um servi�o em nuvem acess�vel aos controladores locais da rede de sensores, de modo a alcan�ar o objetivo (ii) do trabalho. Os t�picos abordados neste cap�tulo englobam o m�todo de troca de dados entre o controlador local e o servidor em nuvem, a interface de controle do usu�rio e a especifica��o de um algoritmo de aprendizagem espec�fico para o dom�nio de controle de ilumina��o.


\section{Desenvolvimento do Algoritmo de Aprendizagem}
Nesta se��o, detalha-se o processo de desenvolvimento de um algoritmo de aprendizagem para o dom�nio de controle de ilumina��o. O algoritmo recebe como entradas um conjunto de dados descrevendo o estado de sensores e atuadores, provenientes do controlador local, e produz como sa�das regras de atua��o, condicionais aos estados dos sensores.

\subsection{Formato de Entrada dos Dados}\label{subsec:formatoentrada}
No dom�nio selecionado para este trabalho, o dado de atua��o relevante seria o estado da l�mpada a ser controlada. O objetivo do algoritmo, ent�o, � prever o valor deste estado baseado nas leituras de outros sensores existentes na rede dom�stica, tais como de luminosidade e presen�a. A Tabela \ref{tab:entrada_learning} ilustra o formato de entrada dos dados.

\begin{table}[h]
	\centering
	\caption{Formato dos dados de entrada para o algoritmo de aprendizagem}\smallskip
	\label{tab:entrada_learning}
	\includegraphics[width=0.5\textwidth]{tabelas/entrada_learning.pdf}
\end{table}

Observe que o problema em quest�o consiste em receber dados de treino, que representam o comportamento do usu�rio, e desenvolver um modelo que preveja o estado do atuador de forma fiel aos dados observados. Trata-se, portanto, de um problema de aprendizagem supervisionada, definido em \cite{james2014} como sendo a gera��o de um modelo que relaciona uma vari�vel-alvo (no caso, o estado da l�mpada) com vari�veis preditoras (presen�a, luminosidade, entre outros). O estado da l�mpada, ent�o, � visto como uma classe associada a cada entrada do conjunto de dados, e o processo de aprendizagem supervisionado que infere esta classe para entradas n�o vistas anteriormente � denominado Classifica��o.

\subsection{Algoritmos de Classifica��o Existentes}\label{subsec:algclass}
Existem diversos algoritmos de classifica��o documentados na literatura, cada qual adotando uma abordagem distinta para gera��o de modelos e classifica��o de dados novos \cite{han2005, james2014}. A seguir ser�o descritos de forma sucinta tr�s t�cnicas candidatas a serem utilizadas no processo de deriva��o de regras para este projeto.

\subsubsection{Indu��o por �rvore de Decis�o}
O modelo de classifica��o gerado por esta t�cnica consiste em uma �rvore de decis�o, em que cada n� interno representa um teste a uma vari�vel preditora, cada aresta representa uma sa�da do teste, e cada n� terminal (folha) indica a classe resultante. A Figura \ref{fig:exemplo_arvore} mostra o exemplo de uma �rvore de decis�o obtida por esta t�cnica. 

Neste exemplo, o conjunto de dados refere-se a consumidores de uma loja de eletr�nicos, e as classes mostradas nos n�s-folha indicam se um consumidor adquire ou n�o certo produto. No caso, uma das regras geradas pelo modelo diz que se o consumidor � jovem e estudante, ent�o ele adquire o produto.

\begin{figure}[h]
	\centering
	\caption{Exemplo de modelo gerado pela t�cnica de Indu��o por �rvore de Decis�o}
  \includegraphics[width=0.8\textwidth]{imagens/exemplo_arvore.png}
  \label{fig:exemplo_arvore}  
  
  Fonte: \cite{han2005}
\end{figure}

Existem diversas implementa��es desta t�cnica de classifica��o, tais como ID3, C4.5 e CART, que adotam uma estrat�gia \textit{greedy} e \textit{top-down} de constru��o da �rvore. Cada uma dessas implementa��es efetua o particionamento dos dados de forma particular, efetuando sele��o de vari�veis por crit�rios tais como ganho de informa��o, raz�o de ganho ou �ndice Gini. As descri��es dessas t�cnicas fogem do escopo deste trabalho, e podem ser encontradas em \cite{han2005}.

\subsubsection{Redes Neurais}
A t�cnica de aprendizagem por redes neurais foi inspirada pelos ramos da psicologia e neurobiologia, que buscaram modelar computacionalmente o comportamento de neur�nios. Uma rede neural � composta por diversas unidades interconectadas arranjadas em camadas, conforme ilustra a Figura \ref{fig:elem_rede_neural}. 

Os dados de entrada do classificador s�o passados para as unidades da camada de entrada. Esses dados, ent�o, s�o combinados linearmente atrav�s de pesos determinados nas interconex�es e passados �s unidades das camadas intermedi�rias, denominadas \textit{hidden layers}. As sa�das da �ltima camada intermedi�ria s�o passadas �s unidades da camada de sa�da, que define a classe resultante.

\begin{figure}[h]
	\centering
	\caption{Elementos de uma rede neural}
  \includegraphics[width=0.8\textwidth]{imagens/elem_rede_neural.png}
  \label{fig:elem_rede_neural}  
  
  Fonte: \cite{han2005}
\end{figure}

O processo de aprendizagem das redes neurais � computacionalmente caro e complexo, envolvendo um processo denominado \textit{backpropagation}. Este � um processo iterativo que consiste em alimentar os dados de treino na rede neural, obter a sa�da com o modelo corrente, e reajustar os pesos do modelo em ordem reversa, partindo das unidades da camada de sa�da.

No entanto, as redes neurais possuem a vantagem de possu�rem alta toler�ncia a dados ruidosos, al�m de n�o requerer conhecimento da rela��o entre as classes a serem previstas e as vari�veis preditoras \cite{han2005}.

\subsubsection{\textit{Support Vector Machines} (SVM)}
O SVM � uma t�cnica de classifica��o que se baseia na defini��o de um hiperplano �timo para efetuar a segrega��o dos dados pertencentes �s diferentes classes. No caso, o hiperplano �timo seria o que prov� maior margem entre os dados, conforme ilustra a Figura \ref{fig:svm_max_margin}. Neste exemplo, o conjunto de dados � bidimensional, e o hiperplano de separa��o � uma reta. Observe que das diversas retas poss�veis mostradas � esquerda, seleciona-se a que resulta em maior margem, mostrada � direita.

\begin{figure}[h]
	\centering
	\caption{Retas candidatas para efetuar a segrega��o dos dados (esq.), e a reta �tima selecionada, que prov� a maior margem no conjunto de dados}
  \includegraphics[width=0.8\textwidth]{imagens/svm_max_margin.png}
  \label{fig:svm_max_margin}  
  
  Fonte: \cite{james2014}
\end{figure}

A t�cnica de SVM se baseia nesta ideia de definir um plano de separa��o �timo, mas efetua a adi��o de mais dimens�es aos dados originais para lidar com situa��es de margens n�o-lineares. Este aumento na dimensionalidade dos dados � feito com base na utiliza��o de \textit{kernels}. A descri��o matem�tica deste processo est� fora do escopo deste trabalho, e pode ser encontrado em \cite{james2014}.

\subsection{Considera��es sobre a Escolha do Algoritmo}
A se��o \ref{subsec:algclass} apresentou tr�s algoritmos de classifica��o pass�veis de serem aplicados no projeto. Conforme mencionado, cada algoritmo possui uma abordagem distinta na constru��o do modelo e na avalia��o de entradas novas para efetuar a classifica��o, possuindo vantagens e desvantagens particulares. Esta se��o lista as considera��es adotadas para a sele��o do algoritmo a ser utilizado na gera��o de regras.

O fator principal utilizado para selecionar o algoritmo de aprendizagem � a interpretabilidade do modelo gerado. Dentre as raz�es para a prioriza��o deste fator, destaca-se a necessidade de o usu�rio ter capacidade de analisar as regras propostas pelo algoritmo, de modo a dar-lhe a escolha de aceit�-la ou recus�-la. Essa possibilidade � extremamente importante, levando-se em conta que os sensores utilizados em ambiente dom�stico podem possuir imprecis�es que resultem na gera��o de regras estatisticamente precisas, mas ainda assim indesejadas pelo usu�rio.

Nesse contexto, \cite{james2014} menciona haver um \textit{tradeoff} entre a flexibilidade e a interpretabilidade dos m�todos de aprendizagem, como mostra a Figura \ref{fig:interpretabilidade_algoritmos}. M�todos flex�veis possuem alta capacidade de gerar modelos que se adaptem a dados de natureza complexa, mas tais modelos acabam sendo de dif�cil interpreta��o pelo usu�rio. 

\begin{figure}[h]
	\centering
	\caption{\textit{Tradeoff} entre flexibilidade e interpretabilidade de m�todos de aprendizagem}
  \includegraphics[width=0.8\textwidth]{imagens/interpretabilidade_algoritmos.pdf}
  \label{fig:interpretabilidade_algoritmos}  
  
  Fonte: \cite{james2014}
\end{figure}

Note que, dentre as t�cnicas apresentadas, redes neurais e SVM possuem alta flexibilidade, permitindo gerar modelos para dados com vari�veis cuja rela��o � desconhecida, a princ�pio (no caso das redes neurais), e para dados com fronteira de decis�o n�o-linear (no caso do SVM). Entretanto, os modelos gerados por estas t�cnicas s�o pouco interpret�veis: redes neurais exp�em um conjunto de pesos que processam os dados, de acordo com a topologia selecionada, e o SVM gera um hiperplano de separa��o.

No extremo oposto encontra-se a t�cnica de classifica��o por �rvores de decis�o. Este m�todo � mais restritivo em termos de flexibilidade, mas gera modelos facilmente interpret�veis. De posse de um modelo como o da Figura \ref{fig:exemplo_arvore}, por exemplo, � intuitivo obter os fatores que influenciam na classifica��o das entradas.

Pelas raz�es mencionadas anteriormente, o algoritmo de classifica��o por �rvore de decis�o foi adotado para derivar regras de atua��o. A seguir ser�o descritos os experimentos feitos com  dados de ilumina��o, englobando o preprocessamento dos dados e a aplica��o do algoritmo de classifica��o propriamente dito.

\subsection{Coleta de Dados}
Conforme mencionado, o dom�nio de aplica��o abordado para concep��o de um algoritmo de aprendizagem de regras � o de controle de ilumina��o. Para iniciar os testes, � necess�rio ter em m�os um \textit{dataset} em formato similar ao descrito na se��o \ref{subsec:formatoentrada}. Para obter tais dados de treino, os membros do grupo adquiriram sensores de presen�a PIR e de ilumina��o, associando-os a um Raspberry Pi, a fim de coletar dados nas respectivas resid�ncias. A Figura \ref{fig:montagem_coleta} mostra as montagens utilizadas para efetuar as coletas nas resid�ncias.

\begin{figure}[h]
	\centering
	\caption{Montagens experimentais para coleta de dados de ilumina��o}
	\smallskip
  \includegraphics[width=0.8\textwidth]{imagens/montagem_coleta.png}
  \label{fig:montagem_coleta}  
\end{figure}

Neste processo, os dados foram amostrados a cada minuto. Em uma das montagens, foi utilizado um bot�o para sinalizar o momento em que o interruptor foi acionado, devido � inviabilidade de conectar o interruptor existente no processo de coleta. Na outra montagem, o estado do interruptor foi inferido posteriormente, de acordo com o hor�rio do dia e da ilumina��o ambiente. O \textit{timestamp}, mostrado na Tabela \ref{tab:entrada_learning}, foi registrado para cada amostra coletada, e consiste na representa��o POSIX do instante de coleta do dado.
\chapter{Testes}\label{chp:testes}
Este capítulo visa descrever o procedimento de teste detalhando as ferramentas utilizadas para testar as partes do projeto individualmente, bem como a integração delas.

O grupo desenvolveu testes unitários de forma intensiva para garantir o correto funcionamento dos módulos do projeto. Para tanto, foram utilizadas algumas ferramentas de automação de testes. No primeiro sub-projeto, o grupo utilizou o Mocha\footnote{Disponível em \url{https://mochajs.org/}}, uma plataforma de execução de testes sobre Node.js (Figura \ref{fig:mocha_teste}). No segundo sub-projeto, duas ferramentas de teste foram utilizadas: o JUnit\footnote{Disponível em \url{http://junit.org/}} e o módulo embutido de testes da linguagem Clojure.

\begin{figure}[h]
	\centering
	\caption{Exemplo de teste unitário utilizando a ferramenta Mocha.}
  \includegraphics[width=0.7\textwidth]{imagens/mocha_teste.png}
  \label{fig:mocha_teste}  
\end{figure}

Os testes de integração foram efetuados desenvolvendo-se versões simplificadas, mas funcionalmente equivalentes, do componente a ser integrado. Como mencionado anteriormente, os módulos do projeto (dispositivos e controlador locais, servidor \textit{web} e aplicativo móvel) foram desenvolvidos de forma independente. Assim, para cada um desses componentes, foi desenvolvido um simulador que possibilitasse integrar com outro módulo a ser testado.

Para os dispositivos e controlador da rede local, foram desenvolvidos versões utilizando o protocolo TCP que podem ser executados no próprio computador; para o servidor, foram desenvolvidos bancos de dados de teste que simulassem o estado atual de uma residência, e para o aplicativo, foi implementada uma versão de depuração capaz de enviar e receber os comandos do protocolo Homecloud (Figura \ref{fig:aplicativo_teste}).

\begin{figure}[h]
	\centering
	\caption{Aplicativo de depuração utilizado nos testes de integração.}
  \includegraphics[width=0.4\textwidth]{imagens/aplicativo_teste.png}
  \label{fig:aplicativo_teste}  
\end{figure}

Em seguida, testes mais avançados foram efetuados, simulando um ambiente de aplicação do sistema. Para tanto, foram desenvolvidos protótipos de sensores, atuadores e controladores locais utilizando computadores Raspberry Pi. O servidor em nuvem, responsável por toda a comunicação entre o controlador local e o aplicativo e pelo processo de aprendizagem de regras, foi hospedado no serviço Amazon Web Services (AWS). Além disso, o grupo desenvolveu um aplicativo de produção utilizando a biblioteca descrita na seção \ref{subsec:lib_app}, de modo a demonstrar as interações possíveis do usuário com o sistema (Figura \ref{fig:app_screenshot}).

\begin{figure}[h]
	\centering
	\caption{Aplicativo de produção para demonstrar as interações do usuário com o sistema.}
  \includegraphics[width=0.8\textwidth]{imagens/app_screenshot.png}
  \label{fig:app_screenshot}  
\end{figure}
\chapter{Conclus�es}\label{chp:conclusoes}
Este cap�tulo visa apresentar uma vis�o cr�tica do grupo acerca dos resultados obtidos neste projeto. A se��o \ref{sec:resultados} apresenta os resultados e contribui��es obtidos neste trabalho, remetendo aos objetivos estabelecidos inicialmente; a se��o \ref{sec:dificuldades} comenta as dificuldades encontradas pelo grupo no decorrer do trabalho; a se��o \ref{sec:trabalhos_futuros} lista itens fora do escopo do presente projeto que podem ser implementados para melhorar sua qualidade e funcionalidade, e a se��o \ref{sec:consideracoes_pessoais} apresenta considera��es pessoais do grupo sobre a disciplina e o est�gio em resid�ncia de \textit{software}.

\section{Resultados e Contribui��es} \label{sec:resultados}
Como resultado do presente trabalho, o grupo conseguiu resolver ou mitigar as limita��es das solu��es existentes de automa��o residencial, conforme descrito na se��o \ref{chp:intro}, atingindo os objetivos do trabalho. Mais especificamente:

\begin{itemize}
	\item O grupo desenvolveu o protocolo Rainfall para ser executado em uma rede de sensores e atuadores local. Este protocolo permite o envio de dados e comandos entre os dispositivos, e prev� a defini��o de regras de automa��o. A especifica��o completa do protocolo foi dada neste documento, e uma implementa��o utilizando a linguagem Node.js est� dispon�vel publicamente no reposit�rio Github;
	
	\item O grupo desenvolveu o protocolo Homecloud, que norteia a comunica��o entre o controlador local, o aplicativo m�vel e o servidor \textit{web}. Este protocolo tamb�m se encontra especificado neste documento, com implementa��es em Node.js (para o controlador), Clojure (para o servidor \textit{web}) e Java/Android (para o aplicativo m�vel) dispon�veis de forma aberta no reposit�rio Github;
	
	\item Por fim, foi documentado e implementado um processo poss�vel de coleta, tratamento e an�lise de dados para fins de deriva��o autom�tica de regras de automa��o. Apesar de o dom�nio de aplica��o ser restrito para controle de ilumina��o, o processo documentado demonstra a viabilidade de se derivar tais regras atrav�s da coleta de dados dos usu�rios e subsequente aplica��o de algoritmos de aprendizagem.
\end{itemize}

\section{Dificuldades} \label{sec:dificuldades}
Devido � natureza extensa do projeto, o grupo encontrou diversos itens de trabalho desafiadores, dos quais destacam-se:

\begin{itemize}
	\item A necessidade de o protocolo Rainfall ser independente dos protocolos de rede ou transporte subjacente. Inicialmente, o grupo planejou projetar o Rainfall para um protocolo de comunica��o subjacente espec�fico ideia esta abandonada por representar uma limita��o importante de sua aplica��o. A solu��o obtida foi inspirada no funcionamento dos protocolos existentes, baseados em camadas independentes;
	
	\item A necessidade de haver comunica��o do servidor em nuvem para os controladores locais. Na arquitetura proposta, o servidor deve ser capaz de enviar mensagens ao controlador local. Isso representa uma dificuldade importante, haja vista que tais controladores, na pr�tica, se encontram em redes com endere�o IP p�blico din�mico e possuem, em geral, um servi�o de NAT intermedi�rio. Este problema foi resolvido atrav�s do estabelecimento de um canal Websocket no momento da identifica��o do controlador com o servidor;
	
	\item O projeto de um algoritmo de aprendizagem gen�rico. Inicialmente, o objetivo do projeto era de projetar um processo para deriva��o de regras de automa��o gen�ricas, aplic�veis a quaisquer dispositivos. Este processo se mostrou muito complexo, haja vista a necessidade de se preprocessar os dados para obter resultados satisfat�rios. O grupo resolver limitar o escopo deste item, provendo gera��o autom�tica de regras para um dom�nio espec�fico (o controle de ilumina��o).
\end{itemize}

\section{Trabalhos Futuros} \label{sec:trabalhos_futuros}
No decorrer deste documento, foram identificadas diversas limita��es e n�o-escopos que comp�em poss�veis trabalhos futuros. Os itens apresentados a seguir incluem a solu��o de tais limita��es, al�m de algumas funcionalidades que o grupo considerou importantes, mas foram mantidos fora da especifica��o original deste trabalho.

\begin{itemize}
	\item Implementa��o de confiabilidade no protocolo Rainfall. O protocolo Rainfall n�o prov� recursos de confiabilidade, dependendo dos protocolos de comunica��o subjacentes para tanto. A implementa��o de mensagens de confirma��o de entrega (ACK) no pr�prio Rainfall permitiria sua utiliza��o de forma confi�vel em protocolos como o UDP;
	\item Implementa��o de criptografia no protocolo Rainfall. O protocolo desenvolvido n�o prov� confidencialidade dos dados trafegados, recurso considerado importante dado � natureza dos dados trafegados na rede;
	\item Suporte a mais de um controlador local. O protocolo foi desenvolvido pensando em uma aplica��o com somente um controlador local operante. Seria interessante estudar as modifica��es necess�rias para prover suporte � opera��o de mais de um controlador, removendo um ponto �nico de falha na rede local. Aspectos a serem estudados incluem a sincroniza��o da base de dados dos controladores, roteamento de pacotes, detec��o de falha dos demais controladores e adapta��o a situa��es de falha.
	\item Suporte a dispositivos legados. O sucesso da proposta deste projeto depende do desenvolvimento de dispositivos que suportem o protocolo proposto. Nesse sentido, seria importante prover suporte a dispositivos legados que n�o suportem o protocolo, permitindo sua integra��o � rede. Uma poss�vel abordagem para este problema seria desenvolver dispositivos de interface que suportem o protocolo Rainfall e traduzam os dados ou comandos em sinais suportados pelo dispositivo legado;
	\item Suporte � previs�o de regras que utilizem grandezas num�ricas. O processo de aprendizagem proposto se restringe a vari�veis categ�ricas, dado o modelo de �rvore de decis�o utilizado. Um poss�vel trabalho futuro seria suportar grandezas num�ricas atrav�s da ado��o de um modelo mais adequado, possivelmente de regress�o;
	\item Adotar um algoritmo de aprendizagem escal�vel. O algoritmo de �rvores de decis�o adotado utiliza um modelo de treinamento em \textit{batch}, baseando-se em todo o conjunto de dados obtidos at� o momento. Seria importante estudar a aplica��o de algoritmos incrementais, que se atualize a cada dado novo recebido, para melhorar a escalabilidade do sistema;
	\item Generalizar o processo de aprendizagem. Possivelmente um dos aspectos mais desafiadores encontrados no projeto � o de generalizar o processo de aprendizagem. Conforme mencionado anteriormente, o grupo teve de fazer um preprocessamento dos dados dependente do dom�nio de aplica��o para obter resultados satisfat�rios. Uma �rea de estudo relevante seria verificar a possibilidade de se aplicar algum processo de treinamento que seja adequado para modelar comportamento humano dependente do tempo.
\end{itemize}

\section{Considera��es Pessoais} \label{sec:consideracoes_pessoais}
O grupo dedicou esta se��o para tecer coment�rios sobre o processo de desenvolvimento deste trabalho, dentro do contexto da disciplina de Trabalho de Formatura e do est�gio de resid�ncia em software.

\subsection{A Disciplina de Trabalho de Formatura}
Por parte da faculdade, os alunos cursam duas disciplinas que se dedicam ao Trabalho de Formatura, com dura��o de um semestre cada. De acordo com o cronograma recebido, o primeiro m�dulo (primeiro semestre) se dedica � defini��o do tema e do documento de especifica��es. O segundo m�dulo, por sua vez, foca na implementa��o e na apresenta��o do progresso em um intervalo mensal.

Na pr�tica, o grupo adotou uma metodologia diferente da proposta. Conforme mencionado na se��o \ref{sec:metodologia}, o desenvolvimento foi efetuado com base em \textit{sprints}, que preveem a disponibiliza��o de produtos tang�veis e verific�veis em intervalos variando de duas semanas a um m�s. Desse modo, o grupo obteve v�rias vers�es funcionais dos elementos do projeto, de forma incremental.

Esta abordagem foi importante para identificar precocemente as limita��es do projeto. Por exemplo, a constata��o da inviabilidade do projeto de um algoritmo de aprendizagem gen�rico se deu em meados do primeiro semestre. Isso deu ao grupo tempo h�bil para comunica��o com os orientadores de modo a ajustar os objetivos e requisitos do projeto.

Uma sugest�o seria incorporar tais princ�pios � disciplina. N�o � raro perceber, pelas apresenta��es parciais, a necessidade dos grupos em efetuar mudan�as no escopo do projeto no meio do segundo semestre, a poucos meses da entrega final do projeto. O grupo acredita fortemente que a metodologia adotada, bem como a execu��o de reuni�es frequentes com os orientadores de modo a apresentar prot�tipos funcionais, foi essencial para evitar imprevistos como os mencionados.

\subsection{O Est�gio em Resid�ncia de Software}
O �ltimo ano da gradua��o � marcado pela necessidade dos alunos de estagiar em empresas. No caso particular do grupo, todos os membros participaram de um programa de est�gio em software nas empresas Evo Systems e Scopus. Um ponto importante desse est�gio foi a possibilidade de focar o trabalho no projeto de formatura.

O est�gio consiste basicamente de reuni�es peri�dicas com funcion�rios da empresa, a quem o grupo apresenta o progresso no trabalho. Nessas reuni�es, s�o obtidos coment�rios, sugest�es e cr�ticas acerca das decis�es j� tomadas e a serem tomadas. Al�m disso, s�o oferecidos cursos com o objetivo de complementar a grade curricular dos alunos, com temas variando em projeto de neg�cio, arquitetura de software e pesquisa aplicada.

O grupo acredita que um est�gio nesses moldes favorece consideravelmente o desenvolvimento de um projeto de formatura de qualidade. A experi�ncia com o presente projeto evidenciou a demanda de tempo necess�ria para o projeto e implementa��o de um sistema de complexidade mediana, e certamente os objetivos deste trabalho n�o poderiam ser atingidos sem a liberdade proporcionada pelo programa de est�gio.

Deste modo, acredita-se que este modelo de est�gio deveria ser expandido. � comum verificar propostas de est�gio que consomem boa parte do tempo h�bil do aluno. Levando em conta que o per�odo de est�gio coincide com o de projeto de formatura e possivelmente com o de outras mat�rias obrigat�rias, � evidente que o trabalho no projeto seja alocado com prioridade reduzida. Assim, um programa de est�gio mais flex�vel e que valorize a forma��o do aluno � fundamental para que se possa dar um enfoque adequado ao projeto.



% ========== Referências ==========
% --- IEEE ---
%	http://www.ctan.org/tex-archive/macros/latex/contrib/IEEEtran
%\bibliographystyle{IEEEbib}

% --- ABNT (requer ABNTeX 2) ---
%	http://www.ctan.org/tex-archive/macros/latex/contrib/abntex2
\bibliographystyle{abntex2-num}

\bibliography{referencias}


% ========== Apêndices (opcional) ==========
%\apendice
%\chapter{}
%\chapter{Beta}


% ========== Anexos (opcional) ==========
%\anexo
%\chapter{Alpha}
%\chapter{}



\end{document}

\chapter{Conclusões}\label{chp:conclusoes}
Este capítulo visa apresentar uma visão crítica do grupo acerca dos resultados obtidos neste projeto. A seção \ref{sec:resultados} apresenta os resultados e contribuições obtidos neste trabalho, remetendo aos objetivos estabelecidos inicialmente; a seção \ref{sec:dificuldades} comenta as dificuldades encontradas pelo grupo no decorrer do trabalho; a seção \ref{sec:trabalhos_futuros} lista itens fora do escopo do presente projeto que podem ser implementados para melhorar sua qualidade e funcionalidade, e a seção \ref{sec:consideracoes_pessoais} apresenta considerações pessoais do grupo sobre a disciplina e o estágio em residência de \textit{software}.

\section{Resultados e Contribuições} \label{sec:resultados}
Como resultado do presente trabalho, o grupo conseguiu resolver ou mitigar as limitações das soluções existentes de automação residencial, conforme descrito na seção \ref{chp:intro}, atingindo os objetivos do trabalho. Mais especificamente:

\begin{itemize}
	\item O grupo desenvolveu o protocolo Rainfall para ser executado em uma rede de sensores e atuadores local. Este protocolo permite o envio de dados e comandos entre os dispositivos, e prevê a definição de regras de automação. A especificação completa do protocolo foi dada neste documento, e uma implementação utilizando a linguagem Node.js está disponível publicamente no repositório Github;
	
	\item O grupo desenvolveu o protocolo Homecloud, que norteia a comunicação entre o controlador local, o aplicativo móvel e o servidor \textit{web}. Este protocolo também se encontra especificado neste documento, com implementações em Node.js (para o controlador), Clojure (para o servidor \textit{web}) e Java/Android (para o aplicativo móvel) disponíveis de forma aberta no repositório Github;
	
	\item Por fim, foi documentado e implementado um processo possível de coleta, tratamento e análise de dados para fins de derivação automática de regras de automação. Apesar de o domínio de aplicação ser restrito para controle de iluminação, o processo documentado demonstra a viabilidade de se derivar tais regras através da coleta de dados dos usuários e subsequente aplicação de algoritmos de aprendizagem.
\end{itemize}

\section{Dificuldades} \label{sec:dificuldades}
Devido à natureza extensa do projeto, o grupo encontrou diversos itens de trabalho desafiadores, dos quais destacam-se:

\begin{itemize}
	\item A necessidade de o protocolo Rainfall ser independente dos protocolos de rede ou transporte subjacente. Inicialmente, o grupo planejou projetar o Rainfall para um protocolo de comunicação subjacente específico ideia esta abandonada por representar uma limitação importante de sua aplicação. A solução obtida foi inspirada no funcionamento dos protocolos existentes, baseados em camadas independentes;
	
	\item A necessidade de haver comunicação do servidor em nuvem para os controladores locais. Na arquitetura proposta, o servidor deve ser capaz de enviar mensagens ao controlador local. Isso representa uma dificuldade importante, haja visto que tais controladores, na prática, se encontram em redes com endereço IP público dinâmico e possuem, em geral, um serviço de NAT intermediário. Este problema foi resolvido através do estabelecimento de um canal Websocket no momento da identificação do controlador com o servidor;
	
	\item O projeto de um algoritmo de aprendizagem genérico. Inicialmente, o objetivo do projeto era de projetar um processo para derivação de regras de automação genéricas, aplicáveis a quaisquer dispositivos. Este processo se mostrou muito complexo, haja vista a necessidade de se preprocessar os dados para obter resultados satisfatórios. O grupo resolver limitar o escopo deste item, provendo geração automática de regras para um domínio específico (o controle de iluminação).
\end{itemize}

\section{Trabalhos Futuros} \label{sec:trabalhos_futuros}
No decorrer deste documento, foram identificadas diversas limitações e não-escopos que compõem possíveis trabalhos futuros. Os itens apresentados a seguir incluem a solução de tais limitações, além de algumas funcionalidades que o grupo considerou importantes, mas foram mantidos fora da especificação original deste trabalho.

\begin{itemize}
	\item Implementação de confiabilidade no protocolo Rainfall. O protocolo Rainfall não provê recursos de confiabilidade, dependendo dos protocolos de comunicação subjacentes para tanto. A implementação de mensagens de confirmação de entrega (ACK) no próprio Rainfall permitiria sua utilização de forma confiável em protocolos como o UDP;
	\item Implementação de criptografia no protocolo Rainfall. O protocolo desenvolvido não provê confidencialidade dos dados trafegados, recurso considerado importante dado à natureza dos dados trafegados na rede;
	\item Suporte a mais de um controlador local. O protocolo foi desenvolvido pensando em uma aplicação com somente um controlador local operante. Seria interessante estudar as modificações necessárias para prover suporte à operação de mais de um controlador, removendo um ponto único de falha na rede local. Aspectos a serem estudados incluem a sincronização da base de dados dos controladores, roteamento de pacotes, detecção de falha dos demais controladores e adaptação a situações de falha.
	\item Suporte a dispositivos legados. O sucesso da proposta deste projeto depende do desenvolvimento de dispositivos que suportem o protocolo proposto. Nesse sentido, seria importante prover suporte a dispositivos legados que não suportem o protocolo, permitindo sua integração à rede. Uma possível abordagem para este problema seria desenvolver dispositivos de interface que suportem o protocolo Rainfall e traduzam os dados ou comandos em sinais suportados pelo dispositivo legado;	
	\item Possibilitar a especificação de regras em que o valor do atuador seja de alguma forma proporcional ao valor dos sensores. Atualmente, as regras são apenas expressões boolenas que envolvem comparação do valor de sensores com valores fixos ou valor de outros sensores e caso alguma expressão booleana seja verdadeira, um valor fixo é atribuído ao atuador. Isso não é muito interessante para atuadores com valor contínuo;
	\item Suporte à previsão de regras que utilizem grandezas numéricas. O processo de aprendizagem proposto se restringe a variáveis categóricas, dado o modelo de árvore de decisão utilizado. Um possível trabalho futuro seria suportar grandezas numéricas através da adoção de um modelo mais adequado, possivelmente de regressão;
	\item Adotar um algoritmo de aprendizagem escalável. O algoritmo de árvores de decisão adotado utiliza um modelo de treinamento em \textit{batch}, baseando-se em todo o conjunto de dados obtidos até o momento. Seria importante estudar a aplicação de algoritmos incrementais, que se atualize a cada dado novo recebido, para melhorar a escalabilidade do sistema;
	\item Generalizar o processo de aprendizagem. Possivelmente um dos aspectos mais desafiadores encontrados no projeto é o de generalizar o processo de aprendizagem. Conforme mencionado anteriormente, o grupo teve de fazer um preprocessamento dos dados dependente do domínio de aplicação para obter resultados satisfatórios. Uma área de estudo relevante seria verificar a possibilidade de se aplicar algum processo de treinamento que seja adequado para modelar comportamento humano dependente do tempo.
\end{itemize}

\section{Considerações Pessoais} \label{sec:consideracoes_pessoais}
O grupo dedicou esta seção para tecer comentários sobre o processo de desenvolvimento deste trabalho, dentro do contexto da disciplina de Trabalho de Formatura e do estágio de residência em software.

\subsection{A Disciplina de Trabalho de Formatura}
Por parte da faculdade, os alunos cursam duas disciplinas que se dedicam ao Trabalho de Formatura, com duração de um semestre cada. De acordo com o cronograma recebido, o primeiro módulo (primeiro semestre) se dedica à definição do tema e do documento de especificações. O segundo módulo, por sua vez, foca na implementação e na apresentação do progresso em um intervalo mensal.

Na prática, o grupo adotou uma metodologia diferente da proposta. Conforme mencionado na seção \ref{sec:metodologia}, o desenvolvimento foi efetuado com base em \textit{sprints}, que preveem a disponibilização de produtos tangíveis e verificáveis em intervalos variando de duas semanas a um mês. Desse modo, o grupo obteve várias versões funcionais dos elementos do projeto, de forma incremental.

Esta abordagem foi importante para identificar precocemente as limitações do projeto. Por exemplo, a constatação da inviabilidade do projeto de um algoritmo de aprendizagem genérico se deu em meados do primeiro semestre. Isso deu ao grupo tempo hábil para comunicação com os orientadores de modo a ajustar os objetivos e requisitos do projeto.

Uma sugestão seria incorporar tais princípios à disciplina. Não é raro perceber, pelas apresentações parciais, a necessidade dos grupos em efetuar mudanças no escopo do projeto no meio do segundo semestre, a poucos meses da entrega final do projeto. O grupo acredita fortemente que a metodologia adotada, bem como a execução de reuniões frequentes com os orientadores de modo a apresentar protótipos funcionais, foi essencial para evitar imprevistos como os mencionados.

\subsection{O Estágio em Residência de Software}
O último ano da graduação é marcado pela necessidade dos alunos de estagiar em empresas. No caso particular do grupo, todos os membros participaram de um programa de estágio em software nas empresas Evo Systems e Scopus. Um ponto importante desse estágio foi a possibilidade de focar o trabalho no projeto de formatura.

O estágio consiste basicamente de reuniões periódicas com funcionários da empresa, a quem o grupo apresenta o progresso no trabalho. Nessas reuniões, são obtidos comentários, sugestões e críticas acerca das decisões já tomadas e a serem tomadas. Além disso, são oferecidos cursos com o objetivo de complementar a grade curricular dos alunos, com temas variando em projeto de negócio, arquitetura de software e pesquisa aplicada.

O grupo acredita que um estágio nesses moldes favorece consideravelmente o desenvolvimento de um projeto de formatura de qualidade. A experiência com o presente projeto evidenciou a demanda de tempo necessária para o projeto e implementação de um sistema de complexidade mediana, e certamente os objetivos deste trabalho não poderiam ser atingidos sem a liberdade proporcionada pelo programa de estágio.

Deste modo, acredita-se que este modelo de estágio deveria ser expandido. É comum verificar propostas de estágio que consomem boa parte do tempo hábil do aluno. Levando em conta que o período de estágio coincide com o de projeto de formatura e possivelmente com o de outras matérias obrigatórias, é evidente que o trabalho no projeto seja alocado com prioridade reduzida. Assim, um programa de estágio mais flexível e que valorize a formação do aluno é fundamental para que se possa dar um enfoque adequado ao projeto.
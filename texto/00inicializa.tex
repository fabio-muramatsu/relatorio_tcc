% documento propriamente monogr�fico: um �nico autor
%\autorPoliI{Nome}{Meio}{Sobrenome}

% documento com dois autores
%\autorPoliII{Nome1}{Meio1}{Sobrenome1}{Nome2}{Meio2}{Sobrenome2}

% documento com tr�s autores
\autorPoliIII{F�bio}{Tsuyoshi}{Muramatsu}{}{Henrique}{Rodrigues}{Ricardo}{Boccoli}{Gallego}

\titulo{Projeto HomeSky}

\orientador{Reginaldo Arakaki}

%\relatFapesp
\monografiaFormatura
%\monografiaMBA
%\qualificacaoMSc{<�rea do Mestrado>}
%\qualificacaoMSc{Enge\-nharia El�trica}
%\dissertacao{<�rea do Mestrado>}
%\qualificacaoDr{<�rea do Mestrado>}
%\teseDr{<�rea do Doutorado>}
%\teseLD
%\memorialLD

%\areaConcentracao{<�rea de Concentra��o>}
\areaConcentracao{Engenharia de Computa��o}

%\departamento{<Departamento>}
\departamento{Departamento de Engenharia de Computa��o e Sistemas Digitais (PCS)}

\local{S�o Paulo}

\data{2016}

\dedicatoria{}

\capa{}

\folhaderosto{}

% Ficha Catalogr�fica

%\setboolean{PoliRevisao}{true} % gera o quadro de revis�o ap�s a defesa
\renewcommand{\PoliFichaCatalograficaData}{%
  1. Assunto \#1. 2. Assunto \#2. 3. Assunto \#3.
  I. Universidade de S�o Paulo. Escola Polit�cnica.
  \PoliDepartamentoData. II. t.}

\fichacatalografica % formata a ficha

\paginadedicatoria{}

\begin{agradecimentos}
\end{agradecimentos}

\begin{resumo}
\end{resumo}

\begin{abstract}
\end{abstract}

%\begin{resume}
%\end{resume}

%\begin{zusammenfassung}
%\end{zusammenfassung}

\tableofcontents

\listoffigures

\listoftables

%\begin{listofabbrv}{1000}
%\item [USP] Universidade de S�o Paulo
%\item [CFS] Courtois-Finiasz-Sendrier
%\end{listofabbrv}
%
%\begin{listofsymbols}{1000}
%\item [$\Delta(h)$] Assinatura di�dica
%\end{listofsymbols}